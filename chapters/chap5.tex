\chapter{検証と評価実験}
この章では本アプリケーションの動作検証と評価実験を行った結果を述べる。
\section{動作検証環境}
動作検証を行った環境を以下に示す。
\begin{itemize}
    \item 機種名:iPad (第9世代)
    \item OS: iPadOS 18.5
\end{itemize}

\section{動作検証}
\subsection{医療・介護・福祉への意志確認機能}
医療・介護・福祉への意志確認機能では、各質問に対し利用者が現在の状況に応じた選択肢を選ぶことで、5.2.3で述べる結果出力機能により、その回答がシートに入力される。
そこで検証にあたり、全質問に対し一律の回答を選択する試行を「いる、いない、多分いる、分からない」の全選択肢について網羅的に行い、各選択肢が正しく機能するかを確認した。
また選択した回答の色の変化や、画面遷移ボタンの色の変化も確認した。\par
以下の図5.1に全選択肢の検証結果を、図5.2に回答選択前と選択後の画面変化を示す。
\begin{figure}[H]
    \centering
    %\includegraphics[width=14cm]{images/chap5/ .png}
    \caption{全選択肢の検証結果}
    \label{fig:all}
\end{figure}

\begin{figure}[H]
    \centering
    %\includegraphics[width=14cm]{images/chap5/ .png}
    \caption{回答選択前と選択後の画面変化}
    \label{fig:select}
\end{figure}

\subsection{属性情報記録機能}
質問回答終了後、出力シートに記載する回答者のニックネームと選択した年代、性別、居住地域を入力する機能を検証した。
以下の図5.3に属性情報の入力時の画面とその出力シートの検証結果を示す。
\begin{figure}[H]
    \centering
    %\includegraphics[width=14cm]{images/chap5/ .png}
    \caption{属性情報の入力時の画面と出力シートの検証結果}
    \label{fig:attribute}
\end{figure}

\subsection{結果出力・保存機能}
結果出力機能では、各質問内容とその回答、属性情報と回答した年月日、最終得点の4項目が出力シートとして表示される機能と
その出力シートを画像ファイルとして保存する機能を検証した。
以下の図5.4に結果出力機能の検証結果を示す。
\begin{figure}[H]
    \centering
    %\includegraphics[width=14cm]{images/chap5/ .png}
    \caption{結果出力機能の検証結果}
    \label{fig:result}

    %\includegraphics[width=14cm]{images/chap5/ .png}
    \caption{結果保存機能の検証結果}
    \label{fig:result2}
\end{figure}


\subsection{履歴確認・共有機能}
履歴確認機能では、保存した過去の出力シートを確認する機能と、その履歴を共有する機能を検証した。
以下の図5.5に履歴確認機能の検証結果を示す。
\begin{figure}[H]
    \centering
    %\includegraphics[width=14cm]{images/chap5/ .png}
    \caption{履歴確認機能の検証結果}
    \label{fig:history}

    %\includegraphics[width=14cm]{images/chap5/ .png}
    \caption{履歴共有機能の共有前の画面}
    \label{fig:history2}

    %\includegraphics[width=14cm]{images/chap5/ .png}
    \caption{履歴共有機能の共有後の別端末の画面}
    \label{fig:history3}
\end{figure}

\section{評価実験}
2025年9月25日に〜代から〜代の男女13人に本アプリケーションを利用してもらい、その後アプリケーションについてのアンケートを実施した。
調査項目は「デザインの見やすさ、就活への意識の向上度、アプリの操作性、アプリの滑らかさ、総合的な満足度」の5項目を
「とても満足、概ね満足、やや不満、とても不満」の4段階に分けて回答を取った。また他にアプリの改善点や意見についての自由記述欄を設けた。
調査の結果を図5.1に示す。
\begin{figure}[H]
    \centering
    \includegraphics[width=14cm]{images/chap5/enquete1.png}
    \caption{評価アンケートの結果(1)}
    \label{fig:enpuete1}
\end{figure}

\begin{figure}[H]
    \centering
    \includegraphics[width=14cm]{images/chap5/enquete2.png}
    \caption{評価アンケートの結果(2)}
    \label{fig:enpuete2}
\end{figure}

\begin{figure}[H]
    \centering
    \includegraphics[width=14cm]{images/chap5/enquete3.png}
    \caption{評価アンケートの結果(3)}
    \label{fig:enpuete3}
\end{figure}

アプリケーションの改善点や課題としては以下の様な点が挙げられた。
\begin{itemize}
    \item 次へボタンの色が淡く、変化したことに気づきにくかった
    \item 戻るや次へのボタンの文字が高齢者には小さい
    \item 紙の方が慣れている
    \item 一人だとできる気がしない
\end{itemize}
