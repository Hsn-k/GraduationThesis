\chapter{検証と評価実験}
この章では本アプリケーションの動作検証と評価実験を行った結果を述べる.
\section{動作検証環境}
動作検証を行った環境を以下に示す.
\begin{itemize}
      \item 機種名:iPad (第9世代)
      \item OS: iPadOS 18.5
\end{itemize}

\section{動作検証}
\subsection{医療・介護・福祉への意志確認機能}
医療・介護・福祉への意志確認機能では,各質問に対し利用者が現在の状況に応じた選択肢を選ぶことで,5.2.3で述べる結果出力機能により,その回答がシートに入力される.
そこで検証にあたり,全質問に対し一律の回答を選択する試行を「いる,いない,多分いる,分からない」の全選択肢について網羅的に行い,各選択肢が正しく機能するかを確認した.\par
以下の図5.1に全選択肢の検証結果を示す.

\begin{figure}[H]
      \centering
      \includegraphics[width=\textwidth]{images/chap5/all.png}
      \caption{全選択肢の検証結果1}
      \label{fig:all}
\end{figure}

\begin{figure}[H]
      \centering
      \includegraphics[width=\textwidth]{images/chap5/all1.png}
      \caption{全選択肢の検証結果2}
      \label{fig:all2}
\end{figure}

\subsection{属性情報記録機能}
質問回答終了後,出力シートに記載する回答者のニックネームと選択した年代,性別,居住地域を入力する機能を検証した.
以下の図5.3に属性情報の入力時の画面とその出力シートの検証結果を示す.
\begin{figure}[H]
      \centering
      \includegraphics[width=\textwidth]{images/chap5/attribute.png}
      \caption{属性情報の入力時の画面と出力シートの検証結果}
      \label{fig:attribute}
\end{figure}

\subsection{結果出力・保存機能・履歴確認機能}
結果出力機能では,各質問内容とその回答,属性情報と回答した年月日,最終得点の4項目が出力シートとして表示される機能と
その出力シートを画像ファイルとして保存する機能,保存した画像を履歴として表示する機能を検証した.
以下の図5.4に結果出力機能の検証結果を示す.
\begin{figure}[H]
      \centering
      \includegraphics[width=\textwidth]{images/chap5/resultsave.png}
      \caption{結果出力・保存機能の検証結果}
      \label{fig:resultsave}
\end{figure}

\begin{figure}[H]
      \centering
      \includegraphics[width=0.5\textwidth]{images/chap5/history.png}
      \caption{履歴確認機能の検証結果}
      \label{fig:history}
\end{figure}

\subsection{履歴共有機能}
履歴共有機能では,保存した過去の出力シートを共有する機能を検証した.
以下の図5.5に履歴共有機能の検証結果を示す.
\begin{figure}[H]
      \centering
      \includegraphics[width=0.6\textwidth]{images/chap5/share.jpg}
      \caption{履歴共有機能の画面}
      \label{fig:share}
\end{figure}

\begin{figure}[H]
      \centering
      \includegraphics[width=\textwidth]{images/chap5/shareDetail.png}
      \caption{履歴共有機能 左図:AirDrop 右図:Mail}
      \label{fig:shareDetail}
\end{figure}

\section{評価実験}
2025年9月25日に30代から80代の男女14人に本アプリケーションを利用してもらい,その後アプリケーションについてのアンケートを実施した.
調査項目は「デザインの見やすさ,就活への意識の向上度,アプリの操作性,アプリの滑らかさ,総合的な満足度」の5項目を
「とても満足,概ね満足,やや不満,とても不満」の4段階に分けて回答を取った.また他にアプリの改善点や意見についての自由記述欄を設けた.
調査の結果を図5.〜に示す.
\begin{figure}[H]
      \centering
      \includegraphics[width=\textwidth]{images/chap5/enquete2.png}
      \caption{評価アンケートの結果(1)}
      \label{fig:enpuete1}
\end{figure}

\begin{figure}[H]
      \centering
      \includegraphics[width=\textwidth]{images/chap5/enquete1.png}
      \caption{評価アンケートの結果(2)}
      \label{fig:enpuete2}
\end{figure}

\begin{figure}[H]
      \centering
      \includegraphics[width=\textwidth]{images/chap5/enquete3.png}
      \caption{評価アンケートの結果(3)}
      \label{fig:enpuete3}
\end{figure}

アプリケーションの改善点や課題として以下が挙げられた.
\begin{itemize}
      \item 次へボタンの色が淡く,変化したことに気づきにくかった
      \item 戻るや次へのボタンの文字が高齢者には小さい
      \item 紙の方が慣れている
      \item 一人だとできる気がしない
\end{itemize}
