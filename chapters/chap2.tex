\chapter{既存のアプリケーションと先行研究}
\section{先行研究}
本章では,医療ソーシャルワーカーの業務におけるデジタル化の現状を整理し,
既存のデジタルソリューションと比較した際の本研究の独自性と新規性を明らかにする.
\par
前節でも述べた通り、医療環境の高度化や複雑化に伴い,医療ソーシャルワーカーにとって業務の円滑かつ迅速な進行は、
より難易度が高まっている切迫した課題となっている.
加えて現在の現場では業務が多様化し,従来の紙ベースのプロセスでは各患者への複雑な要望の対応に限界が生じている.
日本医療ソーシャルワーク学会は, 複雑化する現代の医療課題に対応するため, 医療ソーシャルワーカーの実践力強化を主要な使命として掲げている \cite{jsmsw}.
\par
業務プロセスのデジタル化が期待される一方で,単なる効率化だけでなく,終末期医療における患者自身の意思決定支援など,
医療現場での患者への支援において”人間性”をいかに維持・強化するかが学術的な論点となっている.
特にアドバンス・ケア・プランニングや患者の終末期における意思決定への支援は,医療ソーシャルワーカーが担う最も倫理的であり複雑な分野の一つである.
安井\cite{yasui}は、複数の条件を満たす医療ソーシャルワーカーをエキスパート医療ソーシャルワーカーとし,
彼らが終末期医療に携わる際に生物的・心理的・社会的側面のみならず,スピリチュアルな側面を含めた
全人という存在として,人間を理解する感性を医療ソーシャルワーカー業務の実践の基盤としていることを明らかにした.
この感性は、患者と医療ソーシャルワーカーが共に一人の人間として互いのスピリチュアリティを響き合わせ、苦しみの意味を共に探求するプロセスを支えるものであり、
本アプリケーションのようなデジタル化にあたっては,このような医療ソーシャルワーカーと患者やその家族の間の全人的な対話や関係性を阻害しない設計が必要であると考えられる.
\newpage
\section{既存のデジタルソリューション}
医療ソーシャルワーカーの業務支援に用いることができる既存のデジタルツールは,主に三種類に分類される.
まず一つは業務効率化のためのツールである.
これは前節で述べた課題のうち医療ソーシャルワーカーの業務の円滑かつ迅速な進行を可能にするものである.
代表的なものとして株式会社メドレーが提供する「れんけーさん」が挙げられる\cite{medley}.
これは患者の退院における退院調整業務の負荷軽減を目的としており,転院先の医療機関とのやりとりを簡略化するためのチャット機能や
転院先の病院や介護施設の検索・連携機能を備えている.
\par
次に二つ目は情報共有のためのネットワークである。
これは前節で述べた課題のうち各医療機関との連携において情報の齟齬を生み出さないためのものである.
代表的なものとして千葉県済生会習志野病院が提供する「医療連携ネットワーク」が挙げられる \cite{network}.
これは千葉県済生会習志野病院と連携する近隣の医療機関が、患者の検査結果、レントゲン画像、処方歴などの診療情報の一部を閲覧できる仕組みとなっている。
これにより、近隣の各医療機関がそれぞれで患者の医療情報を管理する必要がなくなるため、情報の齟齬が生まれる可能性を軽減することができる。
しかし、このネットワーク外の医療機関とのやりとりは紙ベースとなってしまう.
\par
最後に三つ目は高齢者向けヘルスケアアプリケーションである。
これは前節で述べた課題のうち、地域住民の医療・介護・福祉に対するリテラシー向上を促進するためのものにあたる。
ソフトバンクの「かんたんHELPO」などの事例は,デジタルリテラシーが必ずしも高くないシニア層向けのヘルスケアアプリケーションである \cite{helpo}.
これは24時間365日利用者が直接医師や看護時、薬剤師のような医療の専門チームとチャットを通してやりとりができる仕組みとなっている。
アプリケーションのチャットから医療機関の受診へとシームレスに繋げられるものであるが、
入院時の医療ケアに対する意思確認や支援者の有無を確認するための機能は提供されていない.
\par
\section{提案するアプリケーション}
上記を踏まえて本研究では以下のような特徴を持つ新しいアプリケーションを提案する.
\begin{itemize}
      \item 実際に用いられている意思確認・支援者の有無確認作業をデジタル化したものである
      \item 各患者の情報をデータとして,いつどこでも管理・参照・共有できるものである
      \item 医療に対する意思や入院時の支援者の有無についての対話を促進するものである
      \item 対話促進にあたって医療機関にかかる回数の多い高齢者向けのUIである
\end{itemize}
ただの医療ソーシャルワーカーの業務支援アプリケーションとしてではなく,UI設計を高齢者向けにするほか,
利用者が本アプリケーションの使用後に医療や入院時についての対話を家族や医療関係者と行うことを促進するものとすることで,
既存のデジタルソリューションとの差異化を図る.