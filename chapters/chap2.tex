\chapter{既存のアプリケーションと先行研究}
\section{先行研究}
本システムと既存システム(エンディングノートアプリ、電子カルテ)との差異は以下の通りである。
第一に、本システムは医療ソーシャルワーカーの既存業務(紙媒体での活動)をアプリ上で効率化・支援するものである。
既存のエンディングノートアプリの多くは、家族構成や遺産相続等の具体的情報の入力・保存を機能の主軸としている。
一方、本システムは患者の終活に対する意識レベルの判定や、不足している視点への助言機能を備えている。
単なる記録媒体ではなく、患者の省察を促し、医療ソーシャルワーカーによる介入を支援する点に独自性がある。
\par
第二に、電子カルテとの差異は「情報の扱われ方」にある。
電子カルテは検査データ等の医療情報を共有するものであるが、本システムは終活に関連する意向や価値観の情報共有に限定している。
医療データと明確に区分することで、患者の意思決定プロセスそのものに焦点を当てた運用を可能にする。
\par

\section{提案するアプリケーション}

