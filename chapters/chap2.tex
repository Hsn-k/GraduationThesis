\chapter{既存のアプリケーションと先行研究}
\section{先行研究}
本システムと既存システム(エンディングノートアプリ、電子カルテ)との差異は以下の通りである。
第一に、本システムは医療ソーシャルワーカーの既存業務(紙媒体での活動)をアプリ上で効率化・支援するものである。
既存のエンディングノートアプリの多くは、家族構成や遺産相続等の具体的情報の入力・保存を機能の主軸としている。
一方、本システムは患者の終活に対する意識レベルの判定や、不足している視点への助言機能を備えている。
単なる記録媒体ではなく、患者の省察を促し、医療ソーシャルワーカーによる介入を支援する点に独自性がある。
\par
第二に、電子カルテとの差異は「情報の扱われ方」にある。
電子カルテは検査データ等の医療情報を共有するものであるが、本システムは終活に関連する意向や価値観の情報共有に限定している。
医療データと明確に区分することで、患者の意思決定プロセスそのものに焦点を当てた運用を可能にする。
\par

%%
\chapter{既存のアプリケーションと先行研究}
\section{先行研究}
本章では、医療ソーシャルワークにおけるデジタル化の現状を整理し、
既存のデジタルソリューションと比較した際の本研究の独自性と新規性を明らかにする。

\section{医療ソーシャルワークにおけるデジタル化の背景}
医療環境の高度化や複雑化に伴い、医療ソーシャルワークの実践力向上は切迫した課題となっている。
日本医療ソーシャルワーク学会は、医療ソーシャルワークの使命を「真の患者支援」と定義しており、これには医療から介護へのシームレスな連携が含まれる[1]。
しかし、現在の現場では業務が多様化し、従来の紙ベースのプロセスでは各患者への複雑な要望の対応に限界が生じている。
プロセスのデジタル化が期待される一方で、単なる効率化だけでなく、終末期医療における患者自身の意思決定支援など、
対人援助における「ヒューマニティ(人間性)」をいかに維持・強化するかが学術的な論点となっている。

\section{既存のデジタルソリューションと市場での標準}
現在の医療・福祉領域におけるデジタルツールは、主に以下の3つの領域に分類される。
\begin{itemize}
\item ワークフロー効率化ツール\par
代表的な先行事例として、株式会社メドレーが提供する「れんけーさん」が挙げられる。
これは退院調整業務の負荷軽減を目的とし、病院や介護施設の検索・連携機能を備えている[7]。
このようなツールは、MSWの事務作業を効率化する「市場標準」としての地位を確立している。
\item 地域医療連携ネットワーク\par
千葉県済生会習志野病院の事例に見られるように、高度なセキュリティを確保した上で診療情報を共有するインフラ構築が進んでいる[9]。
これらは施設間での「情報の非対称性」を解消する上で重要な役割を果たしている。
\item 高齢者向けUX/UI標準\par
ソフトバンクの「かんたんHELPO」などの事例は、デジタルリテラシーが必ずしも高くないシニア層向けのインターフェース設計における基準を示している[10]。
文字サイズの拡大や操作の簡略化は、患者本人や家族が利用するツールにおいて不可欠な要素である。
\end{itemize}

\section{本研究の独自性と解決すべきギャップ}
前述した先行研究や既存システムに対し、本研究が提案するアプリケーションは以下の3点において明確な独自性を有する。

\subsection{データとナラティブの融合} 
既存の電子カルテが「検査データ」などの客観的エビデンスを重視するのに対し、
本システムは患者や家族の「意向」や「価値観」といった**ナラティブ(語り)**の側面に焦点を当てている。
先行研究[6]によれば、アジア圏のACP(アドバンス・ケア・プランニング)においては、単なる選択肢の記録ではなく、
その決定に至る背景を捉えることが重要とされる。本アプリは、MSWが介入するための「対話のきっかけ」を構造化する点に独自性がある。
\subsection{介入支援型の設計} 
既存のエンディングノートアプリの多くは、個人の情報の「蓄積・保存」を主目的としている。
これに対し、本システムは回答結果を「100点満点」で数値化し、準備が不足している分野に対するアドバイスを表示する機能を備えている。
これは、利用者の「自己洞察」を促し、MSWが専門職として介入すべきポイントを可視化する「介入支援型」の設計である。
\subsection{地域分析への応用可能性} 
本システムは、個別のケース管理にとどまらず、収集された属性情報と意思決定状況を紐付けた「地域分析」を視野に入れている。
これは、特定地域における福祉リテラシーの傾向を可視化し、予防的な介入や政策立案の基礎データを提供し得るものであり、既存の汎用的な連携ツールにはない発展性を有している。
%%

\section{提案するアプリケーション}
提案するアプリケーションは、上述した既存システムの課題を克服し、
医療ソーシャルワーカーの専門性を最大限に発揮させるための「対話促進型意思決定支援ツール」である。
本システムは、単なる情報のデジタル化や管理効率化にとどまらず、以下の3つの設計思想に基づいている。
\subsection{専門的知見に基づく「対話の構造化」} 
既存のエンディングノートアプリは、個人の備忘録的な側面が強いが、
本アプリは千葉県済生会習志野病院のMSWが実際に臨床現場で使用している質問項目をベースとしている。
これにより、MSWが患者と向き合う際の「プロの視点」をデジタル上で再現し、
患者自身も気づいていない潜在的な課題(死後事務や具体的な支援者の有無など)を浮き彫りにする。
\subsection{「スコアリング」による意識変容の促進} 
先行研究[6]で強調されているように、終末期医療の意思決定においては、患者の価値観やナラティブ(語り)が重要である。
本アプリは回答を100点満点で数値化し、あえて「得点」としてフィードバックすることで、患者や家族に「何が準備できていて、何が不足しているか」を直感的に認識させる。
この「可視化」のプロセスが、家族間での具体的な話し合い(ACP:アドバンス・ケア・プランニング)を開始する強力なトリガーとなる。
\subsection{MSWの専門性を守るための「効率化」}
メドレーの「れんけーさん」[7]等の既存ツールが施設検索の効率化を主眼に置くのと同様、本アプリも紙媒体による集計・管理の時間を大幅に短縮する。
しかし、その目的は「効率化そのもの」ではなく、生み出された時間をMSWが本来担うべき「複雑な倫理的判断」や「患者の心情に寄り添う相談業務」という、
AIや機械には代替できない専門的業務に充てること(ヒューマニティの維持)にある。