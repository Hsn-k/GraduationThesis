\chapter{結論}
\section{まとめ}
本研究では千葉県済生会習志野病院の協力のもと, 医療ソーシャルワーカーの業務支援および患者とその家族の終活に対する意識向上を目的として, 
意思・支援者有無確認作業をデジタル化するアプリケーションの開発を行った.
開発したアプリケーションは, Google社が提供するフレームワークであるFlutterを採用し, 
iOSおよびAndroidの両OSに対応可能なクロスプラットフォーム開発を行った. 
主な機能として, 「通院」「介護」「逝去時の手続き」「逝去後の事務手続き」の4分野にわたる意思確認機能, 
回答結果の点数化による可視化機能, および結果を画像として保存・共有する機能を実装した. 
これにより, 従来紙媒体で行われていた情報の記録・保管における物理的な制約を解消し, データの管理・共有の利便性を向上させる基盤を構築した.
また, 実装したアプリケーションを用いた評価実験を行った結果, 多くの被験者から操作性やデザインに関して一定の評価を得ることができた. 
特に結果出力機能においては, 自身の回答が可視化されることで終活に対する現状の不足部分や今後の行動指針が明確になり, 
患者や家族間での対話のきっかけ作りに寄与できる可能性が示唆された.

\section{今後の課題}
本研究を通して一定の成果が得られた一方で, 評価実験の結果および運用面において以下の課題が明らかになった.
第一に, ユーザーインターフェースの改善である. 本アプリケーションは高齢者の利用を想定し, ボタンサイズや配色の工夫を行ったが, 
アンケート結果では「ボタンの色が淡く変化に気づきにくい」「文字サイズが小さい」といった指摘が挙げられた. 
特に色覚特性や加齢による視機能の変化をより深く考慮し, WCAG(Web Content Accessibility Guidelines)などの
アクセシビリティ基準に基づいたコントラスト比の調整や, フォントサイズの動的な変更機能の実装が求められる.
第二に, デジタルデバイスへの抵抗感の払拭である. 「紙の方が慣れている」「一人だとできる気がしない」という意見からは, 
デジタル化そのものに対する心理的なハードルが存在することが確認された. これに対し, アプリケーション内での操作ガイドの拡充や, 
医療ソーシャルワーカーが対面でサポートしながら利用することを前提とした「対話補助モード」の搭載など, 
アプリ単体での完結を目指すのではなく, 人的支援とデジタルツールを融合させた運用フローの再設計が必要である.
第三に, データ収集と分析機能の強化である. 現在の実装では, 結果の保存は端末内への画像保存に留まっている. 
本来の目的の一つである「地域分析」や「医療ソーシャルワーカー間での広域な情報共有」を実現するためには, 
クラウドデータベース(Firebase等)との連携機能を追加する必要がある. 
個人情報の取り扱いに十分配慮した上で, 回答データを匿名化してクラウド上に蓄積し, 
地域ごとの終活意識の傾向や支援ニーズを分析できるダッシュボード機能を実装することが今後の重要な展望となる.
これらの課題を解決することで, 本アプリケーションは単なる記録ツールを超え, 
地域医療・福祉連携を支えるより実用的なプラットフォームへと発展することが期待できる.
