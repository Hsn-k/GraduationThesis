\chapter{結論}
\section{まとめ}
本研究では千葉県済生会習志野病院の協力のもと, 医療ソーシャルワーカーの業務支援と患者とその家族における医療・介護・福祉に対する意識向上を目的として,
医療ケアに対する意思確認・入院時における支援者の有無の確認作業をデジタル化するアプリケーションの開発を行った.
開発したアプリケーションは, Google社が提供するフレームワークであるFlutterを採用し,
iOSおよびAndroidの両OSに対応可能なクロスプラットフォーム開発を行った.
主な機能として,入院,介護,死亡時の手続き,死亡後の事務手続きの四つの分野にわたる患者の意思確認機能,
回答内容の出力と画像として保存・管理・共有する機能を実装した.
これにより, 従来紙媒体で行われていた情報の記録・保管における物理的な制約を解消し, データの管理・共有の利便性を向上させる基盤を構築した.
また回答内容を点数として数値で表すことにより達成度を理解しやすくする,ボタンを大きくし,文字よりもイラストを大きく表示させることで視覚的に理解しやすくするなど,
従来紙媒体で行っていた意思確認作業よりも利用者が行いやすく,かつ家族間での対話のきっかけを増やす工夫を施した.
開発した本アプリケーションを用いた評価実験を行った結果,多くの利用者から操作性やデザインに関して一定の評価を得ることができた.
またアンケートの終活に対する意識が向上したかという質問の結果から,多くの利用者が本アプリケーションを通して医療・介護・福祉に対する意識が向上したことが確認できたと言える.
\newpage
\section{今後の課題}
本研究を通して一定の成果が得られた一方で, 評価実験の結果や運用面において以下の三つの課題が明らかになった.
まず第一に, ユーザーインターフェースの改善である.
本アプリケーションの開発にあたっては高齢者の利用を想定し, ボタンサイズや配色の工夫を行ったが,
アンケート結果では"ボタンの色が淡く変化に気づきにくい","文字サイズが小さい"といった点が挙げられた.
特に色覚特性や加齢による視覚機能の変化を考慮し, WCAG(Web Content Accessibility Guidelines)などの
アクセシビリティ基準に基づいたコントラスト比の調整や, フォントサイズの動的な変更機能の実装が必要である.
\par
また第二に, デジタルデバイスへの抵抗感の払拭である.
"紙の方が慣れている","一人だとできる気がしない"という意見からは,
デジタル化そのものに対する心理的なハードルがあることが確認された. これに対し,アプリケーション内での操作ガイドの拡充や,
医療ソーシャルワーカーが対面でサポートしながら利用することを前提とした"対話補助モード"の搭載など,
利用者が一人で,かつアプリ単体での完結を目指すのではなく,医療ソーシャルワーカーないしは家族などの人の支援と
デジタルツールを融合させた形での再設計が必要である.
\par
そして第三に, データ収集と分析機能の強化である.
現在の実装では, 結果の保存は端末内への画像保存に留まっている.
本来の目的の一つである地域分析や医療ソーシャルワーカー間での情報共有を実現するためには,
クラウドデータベースとの連携機能を追加する必要がある.
回答データを匿名化してクラウド上に蓄積し,
地域ごとの終活意識の傾向や支援への要望を分析できるダッシュボード機能を実装することが今後の重要な展望となる.
これらの課題を解決することで, 本アプリケーションは単なる記録ツールではなく,
地域医療・福祉連携を支えるより実用的なプラットフォームへと発展することが期待できる.