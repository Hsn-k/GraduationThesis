\chapter{アプリケーションの内容}
\section{医療・介護・福祉への意志確認機能}
通院や介護,患者が亡くなった際の手続きや亡くなった後の事務手続きについての質問項目に回答することができる.
これらの質問は千葉県済生会病院のソーシャルワーカーが実際に患者やその家族と話し合い,回答してもらうものを用いている.
これらの質問は通院・介護・患者が亡くなった際の手続き・亡くなった後の事務手続きについての4つの分野に分けられており,
利用者がそれぞれを選択することで各分野の質問についての回答を始めることができる.

利用者はアプリケーションを起動し,個人情報の取り扱いに関する同意を行った後,医療・介護・福祉への意志確認画面へと移行する.
本アプリケーションにおける設問内容は,千葉県済生会病院のソーシャルワーカーが実際の臨床現場において,患者およびその家族との対話で使用している質問項目に準拠している.
設問構成は「通院」「介護」「逝去時の手続き」「逝去後の事務手続き」の4領域に体系化されており,利用者は任意の領域を選択することで,各カテゴリにおける意思決定プロセスを開始する設計とした.
また各質問に詳細説明を表示する「詳細」ボタンを実装する.

\begin{figure}[H]
      \centering
      \includegraphics[width=0.8\textwidth]{images/chap3/selectfield.png}
      \caption{分野選択画面}
      \label{fig:SelectField}
\end{figure}
各質問への回答は,いる・いない・多分やってくれる・分からないの4つの選択肢から選択することができる.
またそれぞれの回答に3点・1点・2点・1点の重みをつけ,結果出力の際に回答とともに100点満点換算をした得点が表示される.
\newpage
\section{属性情報記録機能}
各質問に回答した後に,利用者の属性情報を入力する.プライバシー保護の観点から個人を特定しうる情報の収集を回避するため,
氏名はニックネームによる入力を行い,年齢は実数値ではなく年代区分を選択する方式としている.
また,居住地についても詳細な住所は求めず,市町村単位での選択に留めている.

\begin{figure}[H]
      \centering
      \includegraphics[width=0.8\textwidth]{images/chap3/attribute.png}
      \caption{属性情報入力画面}
      \label{fig:Attribute}
\end{figure}
\newpage
\section{結果出力機能}
意思確認機能で選択した回答から点数を算出し,100点満点換算に変換して得点を表示する.
最も得点率の低い分野に適したコメントを得点と共に表示することで,利用者が次にどのようなことについて行動するべきかを明確にした.
また利用者の期待感を高めるために,最終得点を0〜59点,60〜79点,80〜100点と分け,それぞれの範囲で異なるコメントやアニメーションを表示する.

\begin{figure}[H]
      \centering
      \includegraphics[width=0.8\textwidth]{images/chap3/score.png}
      \caption{得点画面}
      \label{fig:Score}
\end{figure}

属性情報記録機能で入力した情報や実施した年月日,医療・介護・福祉への意志確認機能で入力した回答や計算した得点を含めて,
各質問文とともに一覧で確認できるシートを出力することができる.またそのシートを画像として保存することができる.

\begin{figure}[H]
      \centering
      \includegraphics[width=0.8\textwidth]{images/chap3/output.png}
      \caption{結果出力画面}
      \label{fig:Output}
\end{figure}
\newpage
\section{履歴機能}
結果出力機能において保存した画像ファイルを再度表示することができる.
保存された各ファイルは回答した年月日,回答者のニックネームが一覧となって表示され,
そこから該当するファイルを選択することで再度回答結果のシートを確認することができる.
一覧表示は保存した順に表示され,各ファイルの右端に表示されている共有ボタンからメールやAirDropなどで共有することができる.
またゴミ箱の形をしたボタンを押下することで該当ファイルを削除することもできる.

\begin{figure}[H]
      \centering
      \includegraphics[width=0.8\textwidth]{images/chap3/history.png}
      \caption{履歴選択画面}
      \label{fig:History}
\end{figure}
