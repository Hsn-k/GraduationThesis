\chapter{序論}
\section{背景}
%医療機関において,患者やその家族の抱える社会的・経済的・心理的な問題を解決し支える存在として医療ソーシャルワーカーというものがある.
%その中でも病院内に常駐し,患者がどのような医療ケアを受けたいか,終末期に大切にしたいものは何か,
%また終末期の事務手続きや治療の際の付き添いなどをしてくれる支援者はいるのか,ということを
%病院内外のソーシャルワーカー同士や各関係者らと共有し,支援をする存在を院内ソーシャルワーカーと呼ぶ.

医療機関において,患者やその家族の抱える社会的・経済的・心理的な問題を解決し支える存在として医療ソーシャルワーカーというものがある\cite{jaswhs}.
患者の療養中に抱える育児や教育,就労や人間関係への不安を解決するだけでなく,万が一患者が亡くなった場合の家族へのサポートも業務に含まれる.
また患者が転院する際には移る先の病院や介護施設の関係者と連絡をとり患者の援助を行う.退院の場合には患者の職場や学校と連絡をとることで患者の円滑な復職,復学を援助する.

その中でも病院内に常駐し,患者がどのような医療ケアを受けたいか,終末期に大切にしたいものは何か,また終末期の事務手続きや治療の際の付き添いなどをしてくれる支援者はいるのか,ということを
病院内外のソーシャルワーカー同士や各関係者らと共有し,支援をする存在を院内ソーシャルワーカーと呼ぶ.患者やその家族にとって院内ソーシャルワーカーとは,療養中の日常と療養後の日常をつなげる存在と言える.

\section{課題}
従来では上記した患者の意思や支援者の確認は紙媒体で行われてきたが,その情報の共有や保管を紙面のみで行うことは困難だという問題を抱えている.
また病院で勤務する医療ソーシャルワーカーとして地域市民の介護・福祉への意識を高めるための普及活動も行うが,
実際に本人や家族が搬送されるまで医療ケアや周辺手続きについて話し合ったことがないケースも少なくないなど,未だ理解は十分に浸透していない.


\section{目的}
そこで本研究では,千葉県済生会習志野病院の協力のもと,データの管理・共有の利便性向上と,介護・福祉・医療及び死後事務に関する対話のきっかけを増やすことを目的として,
現在習志野病院において実施されている意思・支援者有無確認作業をデジタル化するアプリケーションの開発を行う.
また上記の二つの活動に利用するために,スマートフォンやタブレットからの情報を収集し,医療ソーシャルワーカー間での情報共有や,それを用いた地域分析を行う.
また実証として、医療ソーシャルワーカーや高齢者やその家族にiPadを用いて実際に本アプリケーションを利用してもらい、利用後のアンケートの回答から共有の利便性向上、対話のきっかけが増加したかを確認する。

\section{本論文の構成}
本論文は以下の構成になっている.
〜〜〜