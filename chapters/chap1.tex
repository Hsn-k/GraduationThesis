\chapter{序論}
\section{背景}
現代の医療現場において、医療ソーシャルワーカー(Medical Social Worker)は、
患者とその家族が直面する社会的・経済的・心理的な課題を解決し、その生活を支える不可欠な存在である [1]。
患者は疾病の治療過程において、身体的な苦痛のみならず、育児や教育、就労の継続、あるいは複雑な人間関係といった多面的な不安を抱えることが多い。
医療ソーシャルワーカーはこれらの心理社会的課題に対し、面接や環境調整を通じて介入し、不安の軽減を図る。
さらにその支援は、患者自身の生存中にとどまらず、万が一患者が逝去した場合の遺族に対する死別後のケアや、法的手続きのサポートまで包括的に行われる。
また、医療ソーシャルワーカーは「療養の場の移行」における調整役としての機能も担う。
転院が必要な際には、転院先の医療機関や介護施設の担当者と密に連携し、患者の状態に適した環境への円滑な移行を支援する。
自宅への退院が可能となった場合には、職場や学校等の所属機関と連絡調整を行い、復職や復学といった社会復帰が円滑に進むよう環境を整備する。
このように、医療ソーシャルワーカーは院内にとどまらず、地域社会や関係機関をつなぐコーディネーターとしての役割を果たしている。
広義の医療ソーシャルワーカーの中でも、特に病院内に常駐し、患者の価値観に深く寄り添う役割を担う者を、
本稿では「院内ソーシャルワーカー」と定義し、その専門性を強調したい。
以上から、患者やその家族にとって院内ソーシャルワーカーとは、単なる事務的な調整役ではない。
彼らは、入院という非日常的な「療養中の日常」と、退院後や終末期を含めた「療養後の日常」を断絶させることなく、
シームレスにつなぎ合わせる架け橋としての役割を果たしていると言える。
\par
\section{課題}
従来、患者の意思決定内容や支援者に関する情報の記録・確認は、主に紙媒体を用いて行われてきた。
しかし、現代の高度で複雑化したチーム医療において、紙媒体のみに依存した情報管理は限界を迎えつつある。
具体的には、物理的な保管場所の制約に加え、多職種間でのリアルタイムな情報共有が困難である点が挙げられる。
紙カルテやアナログな記録媒体では、緊急時や時間外対応において必要な情報への迅速なアクセスが阻害されるリスクがある。
また、患者の意向が変化した際の履歴管理や、院内外の関係機関とのシームレスな連携においても、情報の非対称性を生む要因となっている。
また、地域社会に対する課題も深刻である。医療ソーシャルワーカーは、院内業務と並行して、地域住民の医療・介護・福祉に対するリテラシー向上を目的とした普及啓発活動に従事している。
しかしながら、その重要性が地域社会の隅々まで十分に浸透しているとは言い難いのが現状である。
実際、本人が救急搬送されたり、病状が急変したりといったクリティカルな局面を迎えて初めて、家族や関係者が具体的な医療ケアの方針や、
それに伴う事務手続きについて話し合うという事例が後を絶たない。
平時からの対話が未実施のまま緊急事態に直面するケースは少なくなく、啓発活動と実際の行動変容(事前準備)の間には大きな乖離が存在していると言える。
\par
\section{目的}
前節で述べた「情報管理の煩雑さ」と「対話機会の欠如」という課題を解決するため、
本研究では千葉県済生会習志野病院の協力のもと、ICTを活用した支援システムの構築と検証を行う。本研究の主たる目的は以下の2点である。
\begin{itemize}
      \item データ管理\par
            共有の効率化: 従来紙媒体で行われていた意思確認および支援者情報の記録をデジタル化し、アクセシビリティと保存性を向上させる。
      \item 対話の活性化\par
            アプリケーションの利用プロセスを通じて、介護・福祉・医療ケア、さらには死後事務に関する患者・家族間の対話のきっかけを創出し、意識変容を促す。
\end{itemize}
\par
上記目的を達成するため、現在同病院において運用されている「意思・支援者有無確認」のワークフローをデジタル化するアプリケーションを開発する。
本システムは、タブレット端末(iPad等)での利用を前提とし、場所を選ばずに情報の入力・閲覧を可能にする。
収集されたデータは一元管理され、医療ソーシャルワーカー間でのリアルタイムな情報共有を実現する。
さらに、本システムは単なる個別ケースの管理にとどまらず、蓄積されたデータを活用した「地域分析」への応用も視野に入れる。
地域ごとの支援ニーズや意識傾向を可視化することで、病院から地域社会へのより効果的な働きかけや、予防的な介入施策の立案に寄与する基盤を構築する。


%医療機関において,患者やその家族の抱える社会的・経済的・心理的な問題を解決し支える存在として医療ソーシャルワーカーというものがある\cite{jaswhs}.
%患者の療養中に抱える育児や教育,就労や人間関係への不安を解決するだけでなく,万が一患者が亡くなった場合の家族へのサポートも業務に含まれる.
%また患者が転院する際には移る先の病院や介護施設の関係者と連絡をとり患者の援助を行う.退院の場合には患者の職場や学校と連絡をとることで患者の円滑な復職,復学を援助する.

%その中でも病院内に常駐し,患者がどのような医療ケアを受けたいか,終末期に大切にしたいものは何か,また終末期の事務手続きや治療の際の付き添いなどをしてくれる支援者はいるのか,ということを
%病院内外のソーシャルワーカー同士や各関係者らと共有し,支援をする存在を院内ソーシャルワーカーと呼ぶ.患者やその家族にとって院内ソーシャルワーカーとは,療養中の日常と療養後の日常をつなげる存在と言える.

%\section{課題}
%従来では上記した患者の意思や支援者の確認は紙媒体で行われてきたが,その情報の共有や保管を紙面のみで行うことは困難だという問題を抱えている.
%また病院で勤務する医療ソーシャルワーカーとして地域市民の介護・福祉への意識を高めるための普及活動も行うが,
%実際に本人や家族が搬送されるまで医療ケアや周辺手続きについて話し合ったことがないケースも少なくないなど,未だ理解は十分に浸透していない.


%\section{目的}
%そこで本研究では,千葉県済生会習志野病院の協力のもと,データの管理・共有の利便性向上と,介護・福祉・医療及び死後事務に関する対話のきっかけを増やすことを目的として,
%現在習志野病院において実施されている意思・支援者有無確認作業をデジタル化するアプリケーションの開発を行う.
%また上記の二つの活動に利用するために,スマートフォンやタブレットからの情報を収集し,医療ソーシャルワーカー間での情報共有や,それを用いた地域分析を行う.
%また実証として、医療ソーシャルワーカーや高齢者やその家族にiPadを用いて実際に本アプリケーションを利用してもらい、利用後のアンケートの回答から共有の利便性向上、対話のきっかけが増加したかを確認する。

\section{本論文の構成}
本論文は以下の構成になっている.
第一章では、本研究の背景と課題、目的について述べる。\par
第二章では、既存のアプリケーションと先行研究について、またそれを踏まえて提案するアプリケーションについて述べる。\par
第三章では、本アプリケーションの内容について述べる。\par
第四章では、本アプリケーションの実装方法について述べる。\par
第五章では、本アプリケーションの動作検証と評価実験について述べる。\par
最後に第六章では、本研究についての結論と考察を述べる。\par