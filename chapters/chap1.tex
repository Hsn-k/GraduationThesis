\chapter{序論}
\section{背景}
現代の医療現場において,医療ソーシャルワーカー(Medical Social Worker)とは,
患者とその家族が直面する社会的・経済的・心理的な課題を解決し,その生活を支える不可欠な存在である \cite{jaswhs}.
患者は病気の治療過程において,身体的な苦痛だけではなく,育児や教育,仕事や人間関係への影響などの様々な不安を抱えることが多い.
医療ソーシャルワーカーはこれらの課題に対して,面接や環境調整を通じて介入し,不安の軽減を図るという役割を持つ.
さらにその支援は,患者自身の生存中だけではなく,万が一患者が逝去した場合の遺族に対するケアや,法的手続きのサポートまで多岐にわたって行われる.
\par
また,医療ソーシャルワーカーは患者の転院や退院,在宅医療への移行における調整役としての機能も担う.
転院が必要な際には,転院先の医療機関や介護施設の担当者と密に連携し,より患者の状態に適した環境への円滑な移行を支援する.
退院する場合や在宅医療が可能となった場合には,上記した医療関係者とのやりとりだけではなく,
職場や学校等の所属機関とも連絡調整を行い,復職や復学といった社会復帰が円滑に進むよう環境を整備する.
このように,医療ソーシャルワーカーは院内にとどまらず,地域社会や関係機関をつなぐコーディネーターとしての役割を果たしている.
\par
広義の医療ソーシャルワーカーの中でも,特に病院内に常駐し,患者の価値観に深く寄り添う役割を担う者を,
本稿では"院内ソーシャルワーカー"と定義する.
患者やその家族にとって院内ソーシャルワーカーとは,単なる事務的な調整役ではなく,
彼らは,入院という非日常である療養期間と,退院後や逝去してしまった後にも続く日常を断絶させることなく,
シームレスにつなぎ合わせる架け橋としての役割を果たしていると言える.
\par
\newpage
\section{課題}
従来,患者の医療に対する意思決定や療養期間中の支援者に関する情報の記録や確認は,主に紙媒体を用いて行われてきた.
しかし,現代の高度で複雑化したチーム医療において,紙媒体のみでの情報管理は限界を迎えつつある.
\par
具体的には,物理的な保管場所の制約に加え,多職種間でのリアルタイムな情報共有が困難である点が挙げられる.
紙カルテなどのアナログな記録媒体では,緊急時や時間外対応において必要な情報への迅速なアクセスが難しいという問題点がある.
また,患者の医療に対する意思が変化した際の履歴管理や,院内外の関係機関との連携においても,情報の齟齬を生む要因となってしまう.
\par
加えて地域社会に対する課題も深刻である.
医療ソーシャルワーカーは,院内業務と並行して,地域住民の医療・介護・福祉に対するリテラシー向上を目的とした普及啓発活動を行っている.
しかし,その重要性が地域社会の隅々まで十分に浸透しているとは言い難いというのが現状である.
実際,本人が救急搬送されたり,病状が急変したりといった緊急性を要する状況になって初めて,
家族や関係者が具体的な医療ケアの方針や,それに伴う事務手続きについて話し合うという事例が後を絶たない.
日常生活の中での対話が未実施のまま緊急事態に直面するケースは少なくなく,啓発活動と実際の患者の事前準備の間には大きな乖離が存在していると言える.
\par
\section{目的}
前節で述べた”情報管理の煩雑さ”と”対話機会の欠如”という課題を解決するため,
本研究では千葉県済生会習志野病院の協力のもと,ICTを活用した医療ソーシャルワーカー業務の支援システムアプリケーションの構築と検証を行う.
本研究の主な目的はデータ管理化と対話の活性化の二つである.
\par
まず一つ目のデータ管理化として,患者の医療に対する意思確認,および患者の療養中の支援者の情報の利便性と保存性を向上させるため,
従来では紙を用いて行われている作業をデジタル化する.
また二つ目の対話の活性化は,複数ある医療ソーシャルワーカー業務のうち,
地域住民の医療・介護・福祉に対するリテラシー向上を目的とした普及啓発活動の支援に当たるものである.
本研究で開発するアプリケーションの利用を通して,介護・福祉・医療ケア,さらには死後事務に関する患者・家族間の対話のきっかけを作ることで,
医療関係者ではない,いわゆる一般市民である人々に対して医療ケアへの意識を促す.
また対話の活性化の判断基準として,本アプリケーションの開発後に実際に一般の人々に利用してもらい,
アプリケーションの使用感と医療ケアに対する意識変化についてのアンケートを実施する.
\par
上記目的を達成するため,現在同病院において運用されている意思・支援者有無確認のワークフローをデジタル化するアプリケーションを開発する.
本研究において本システムはiPadでの利用を前提とし,場所を選ばずに情報の入力・閲覧を可能にする.
収集されたデータは一元管理され,医療ソーシャルワーカー間でのリアルタイムな情報共有を実現する.
さらに,本システムは単なる個別ケースの管理にとどまらず,蓄積されたデータを活用した”地域分析”への応用も視野に入れる.
地域ごとの支援に対する要望や意識の傾向を可視化することで,病院から地域社会へ対するより効果的な働きかけを支援する.
\section{本論文の構成}
本論文は以下の構成になっている.
第一章では,本研究の背景と課題,目的について述べる.\par
第二章では,既存のアプリケーションと先行研究について,またそれを踏まえて提案するアプリケーションについて述べる.\par
第三章では,本アプリケーションの内容について述べる.\par
第四章では,本アプリケーションの実装方法について述べる.\par
第五章では,本アプリケーションの動作検証と評価実験について述べる.\par
最後に第六章では,本研究についての結論と考察を述べる.\par