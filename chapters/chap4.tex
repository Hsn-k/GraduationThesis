\chapter{実装}
\section{フレームワークの選定}
本アプリケーションを開発するにあたり、Google社が開発したモバイルアプリケーション向けフレームワークであるFlutterを用いた。
採用した理由として以下が挙げられる。
\begin{itemize}
  \item クロスプラットフォーム対応\par
        Flutterでは、Google社が開発したプログラミング言語であるDart言語を使用して開発を行うことができる。
        このDart言語を用いることで単一のコードをベースとしてiOSとAndroidを対象として同時に開発をすることができる。
        \par
        本研究では実証実験においてiOSのタブレット端末を利用するためiOS端末を対象としたが、
        2025年9月スマートフォンOSシェア調査\cite{mobileos}によると、iOSが48.3\%、Androidが51.4\%とあり拮抗している結果となっている。
        また同調査において、高齢者を含む60代の主に利用しているスマートフォンのOSのAndroid利用率が男性では65.0\%、女性では61.9\%とあり、
        他世代に比べAndroidの利用率が多いことが分かる。
        一方で2024年暦タブレット端末国内出荷台数調査\cite{tabletos}によると、iOS端末を出荷しているAppleが2010年から15年連続でシェア率一位を獲得しており、
        2024年暦年のシェア率では51.6\%と、Androidのシェア率を上回る結果となっている。
        以上より利用者への有用性を考慮するとiOSとAndroidの両方に対応する必要があるため、クロスプラットフォームでの開発が望ましいと考えた。
        \par
  \item パッケージの豊富さ\par
        flutterには多くのパッケージが提供されており、本アプリケーションのデータ管理や画像保存などの機能の開発を効率化するだけでなく、将来的な機能拡張を行う際にも有用であると判断した。
        本アプリケーションの機能開発において主に利用したパッケージを以下に挙げる\par
        \begin{itemize}
          \item flutter\_riverpod 2.5.1\par
          \item screenshot 3.0.0\par
          \item flutter\_path\_provider 2.1.4\par
        \end{itemize}
\end{itemize}

\section{開発環境}
本アプリケーションの開発にはVisual Studio Codeを用いた。またXcodeを用いてエミュレータでの動作確認を行った。
\begin{itemize}
  \item Mac Book Pro Intel Core i5
  \item MacOS Ventura 13.7.8
  \item Flutter 3.29.3
  \item Dart 3.7.2
  \item Visual Studio Code 1.106.2
  \item Xcode 15.0
\end{itemize}

\section{機能実装}
\subsection{医療・介護・福祉への意志確認機能}
実装において、各分野を選択するQuestionHomePageクラスと各質問を表示するTrtPageクラスの二つを作成した。
分野選択画面では各分野がどのような状況か分かりやすくするため、各分野をカード形式に配置し文字よりも画像を大きく表示させた。
またgo\_routerを用いることで、各分野のボタンを押下するとcategoryIndexによって該当する質問ページへ遷移する仕組みになっている。
各分野の回答状況は、flutter\_riverpodを用いてcategoryCompletionProviderに格納することで管理している。
その回答状況を参照した結果回答済みの分野はグレーで表示し、回答していない分野は通常の色で表示することで、回答済みか未回答かを視認しやすくした。
また画面下部の「次へ進む」ボタンは、全ての分野が回答済みである場合のみグレーから青色に変化し、押下できるようにすることで回答漏れを完全に防げる仕様とした。
\par
各質問を表示し回答内容を保存する機能でも、flutter\_riverpodを用いることで回答内容とその点数を管理している。
回答状況はそれぞれtrtSelectedList、回答の得点はtrtScoreListProviderにそれぞれの情報を格納することで管理している。
各質問への回答は、いる・いない・多分やってくれる・分からないの4つの選択肢にそれぞれ3点・1点・2点・1点の重みを設定した。
利用者が選んだ選択肢の色を緑色に変更することで、どの選択肢が選ばれているのかを視認しやすくした。
\par
また医療や法律に関する専門用語や質問内容の補足説明を表示するため,Infoクラスを作成しflutter\_materialのshowdialogを用いることで「詳細」ボタンを押下するとダイアログが表示される仕様を実現した。
各詳細内容を記述したsubtitleListを作成し、各質問に振り分けられたtrtNumという番号を対応させることで、該当する説明文を表示させる仕組みとなっている。
\par

\subsection{属性情報記録機能}
この機能では利用者のニックネーム、年代、性別、居住地、そして回答日の5項目を記録する。各項目について入力や選択される度に対応するProviderの状態が更新されることで、選択状態を表現している。
回答日はDateTime.now()を用いて取得した。年代や性別、居住地のように選択肢を選ぶものについては、前述した医療・介護・福祉への意志確認機能と同様に、flutter\_riverpodを用いて管理している。
年代や性別、居住地はそれぞれageAns,genderAns,regionAnsというProviderに格納される。性別や居住区の選択では選択肢が少ないためToggleButtonsとisSelectedを用いていたが、
年代の選択においてはボタンを画面内に収めるため改行できるようにWrapとChoiceChipを用いている。
\par
ニックネームもflutter\_riverpodを用いて管理しており、文字列はnameAnsというProviderに格納される。
ニックネームの入力にあたってはFocusを使用することで入力欄からフォーカスが外れたタイミングで再描画が行われる仕様になっている。
そのため一文字ごとの再描画が行われないことによりアプリケーションの動作軽量化を図っている。
\par
また画面下部に結果出力機能へ遷移するボタンを配置してあるが、そのボタンが押下された時に全ての項目に対して各Providerの状態を確認するようになっている。
全ての項目が正しく入力、選択されている場合のみ結果表示画面へ遷移が可能となり、未入力のまま結果が出力されることを防いでいる。

\subsection{結果出力機能}



\subsection{履歴確認・共有機能}



\section{画面遷移}
以下に本アプリケーションの画面遷移図を示す。
\begin{figure}[H]
  \centering
  %\includegraphics[width=14cm]{images/mockup.png} % 画像がないためコメントアウト
  \caption{画面遷移図}
  \label{fig:mockup}
\end{figure}
アプリケーションを起動することでホーム画面が表示される。
ホーム画面には本アプリケーションの名称とメインビジュアル、スタートボタン、履歴表示画面へ遷移するボタンが配置されている。
スタートボタンを押下すると、属性情報の取り扱いについての同意書が表示される。
同意書の画面下部に同意するか否かのボタンが配置され、同意しないボタンを押下するとホーム画面に戻る。同意するボタンを押下すると回答者の属性選択画面に遷移する。
属性選択画面では、回答者の属性を選択することで質問分野選択画面に遷移する。各分野のボタンを押下することで、その分野の質問への回答が始まる。
\par
各質問の画面下部には、分野の回答進行度を示すバー、前の質問に戻るボタン、次の質問に進むボタンが配置されており、回答を選択することで次の質問に進むボタンが押下できる様になる。
質問文の右下には詳細ボタンが配置されており、押下することでその質問に対する詳しい内容が表示される。詳細文が表示されているボックスの右上にあるばつ印を押下することで表示を閉じることができる。
全ての質問に対して回答が終わると分野選択画面の下部に配置されている、次へ進むボタンが押下できる様になり属性情報記録画面へと遷移する。
\par
属性情報記録画面では全ての属性情報を入力、選択をすることで結果発表画面に遷移することができる。
結果発表画面の下部にある次へ進むボタンを押下することで結果出力・共有画面へと遷移する。
利用者はホーム画面と属性情報取り扱い同意書の画面以外の画面上部にあるホームボタンを押下することでホーム画面に遷移することができる。

\section{UIの実装}
本アプリケーションの主要な対象者として、医療ソーシャルワーカーと通院頻度が高くソーシャルワーカーの支援を必要としやすい高齢者とその家族を想定している。
そのため高齢者の加齢に伴う視力の低下に配慮し、ボタンやテキストサイズは一般的なアプリケーションよりも大きく設計することで視認性を確保している。
操作ボタンについては、入力状況に応じて色を変化させることで操作の可否を直感的に判別可能にした。
また視覚的な補助として画像やピクトグラムを大きく配置し、テキストの判読が困難な場合でも操作意図が伝わるよう操作性の向上を図った。
