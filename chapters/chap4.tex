\chapter{実装}
\section{フレームワークの選定}
本アプリケーションを開発するにあたり,Google社が開発したモバイルアプリケーション向けフレームワークであるFlutterを用いた.
採用した理由として以下が挙げられる.
\begin{itemize}
      \item クロスプラットフォーム対応\par
            Flutterでは,Google社が開発したプログラミング言語であるDart言語を使用して開発を行うことができる.
            このDart言語を用いることで単一のコードをベースとしてiOSとAndroidを対象として同時に開発をすることができる.
            \par
            %この文章の位置??ここ???箇条書きの中だからか段落がつかないのがキモい
            本研究では実証実験においてiOSのタブレット端末を利用するためiOS端末を対象としたが,
            2025年9月スマートフォンOSシェア調査\cite{mobileos}によると,iOSが48.3\%,Androidが51.4\%とあり拮抗している結果となっている.
            また同調査において,高齢者を含む60代の主に利用しているスマートフォンのOSのAndroid利用率が男性では65.0\%,女性では61.9\%とあり,
            他世代に比べAndroidの利用率が多いことが分かる.
            一方で2024年暦タブレット端末国内出荷台数調査\cite{tabletos}によると,iOS端末を出荷しているAppleが2010年から15年連続でシェア率一位を獲得しており,
            2024年暦年のシェア率では51.6\%と,Androidのシェア率を上回る結果となっている.
            以上より利用者への有用性を考慮するとiOSとAndroidの両方に対応する必要があるため,クロスプラットフォームでの開発が望ましいと考えた.
            %この文章の位置??ここ???箇条書きの中だからか段落がつかないのがキモい
            \par
      \item パッケージの豊富さ\par
            Flutterには多くのパッケージが提供されており,本アプリケーションのデータ管理や画像保存などの機能の開発を効率化するだけでなく,将来的な機能拡張を行う際にも有用であると判断した.
            本アプリケーションの機能開発において主に利用したパッケージを以下に挙げる\par
            \newpage
            \begin{itemize}
                  \item Flutter\_riverpod 2.5.1\par
                        Flutter\_riverpod\cite{riverpod}とは,状態管理のためのパッケージである.
                        提供されているproviderクラスを用いることで,データ保存を行うことができる.\par
                  \item screenshot 3.0.0\par
                        Flutter\_screenshot\cite{screenshot}とは,ウィジェットを撮影し画像として保存するためのパッケージである.
                        本アプリケーションでは,回答内容を画像として保存する際に使用した.\par
                  \item Flutter\_path\_provider 2.1.4\par
                        Flutter\_path\_provider\cite{pathprovider}とは,デバイスのファイルのパスを取得するためのパッケージである.
                        上記のFlutter\_screenshotパッケージを用いて撮影した画像を保存する際に,ファイルの保存先を指定するためのパッケージとして使用する.
                        また,アプリが実行されているプラットフォームに応じて適切なディレクトリが選択できるため,将来的な機能拡張を行う際にも有用である.\par
            \end{itemize}
\end{itemize}

\section{開発環境}
本アプリケーションの開発にはVisual Studio Codeを用いた.またXcodeを用いてエミュレータでの動作確認を行った.
\begin{itemize}
      \item Mac Book Pro Intel Core i5
      \item MacOS Ventura 13.7.8
      \item Flutter 3.29.3
      \item Dart 3.7.2
      \item Visual Studio Code 1.106.2
      \item Xcode 15.0
\end{itemize}
\section{機能実装}
\subsection{医療・介護・福祉への意志確認機能}
%画面遷移について
実装において,各分野を選択するQuestionHomePageクラスと各質問を表示するTrtPageクラスの二つを作成した.
分野選択画面では各分野がどのような状況か分かりやすくするため,各分野をカード形式に配置し文字よりも画像を大きく表示させた.
またgo\_routerを用いることで,各分野のボタンを押下するとcategoryIndexによって該当する質問ページへ遷移する仕組みになっている.
各分野の回答状況は,Flutter\_riverpodを用いてcategoryCompletionProviderに格納することで管理している.
その回答状況を参照した結果回答済みの分野はグレーで表示し,回答していない分野は通常の色で表示することで,回答済みか未回答かを視認しやすくした.
また画面下部の「次へ進む」ボタンは,全ての分野が回答済みである場合のみグレーから青色に変化し,押下できるようにすることで回答漏れを完全に防げる仕様とした.
\par
%得点と回答内容の保存について
各質問を表示し回答内容を保存する機能でも,Flutter\_riverpodを用いることで回答内容とその点数を管理している.
回答状況はそれぞれtrtSelectedList,回答の得点はtrtScoreListProviderにそれぞれの情報を格納することで管理している.
各質問への回答は,いる・いない・多分やってくれる・分からないの4つの選択肢にそれぞれ3点・1点・2点・1点の重みを設定した.
利用者が選んだ選択肢の色を緑色に変更することで,どの選択肢が選ばれているのかを視認しやすくした.
\par
%詳細ボタンについて
また医療や法律に関する専門用語や質問内容の補足説明を表示するため,
Infoクラスを作成しFlutter\_materialのshowdialogを用いることで「詳細」ボタンを押下するとダイアログが表示される仕様を実現した.
各詳細内容を記述したsubtitleListを作成し,各質問に振り分けられたtrtNumという番号を対応させることで,該当する説明文を表示させる仕組みとなっている.
\par

\subsection{属性情報記録機能}
%ボタンについて
この機能では利用者のニックネーム,年代,性別,居住地,そして回答日の5項目を記録する.各項目について入力や選択される度に対応するProviderの状態が更新されることで,選択状態を表現している.
回答日はDateTime.now()を用いて取得した.年代や性別,居住地のように選択肢を選ぶものについては,前述した医療・介護・福祉への意志確認機能と同様に,Flutter\_riverpodを用いて管理している.
年代や性別,居住地はそれぞれageAns,genderAns,regionAnsというProviderに格納される.性別や居住区の選択では選択肢が少ないためToggleButtonsとisSelectedを用いていたが,
年代の選択においてはボタンを画面内に収めるため改行できるようにWrapとChoiceChipを用いている.
\par
%名前入力について
ニックネームもFlutter\_riverpodを用いて管理しており,文字列はnameAnsというProviderに格納される.
ニックネームの入力にあたってはFocusを使用することで入力欄からフォーカスが外れたタイミングで再描画が行われる仕様になっている.
そのため一文字ごとの再描画を行わないことによりアプリケーションの動作軽量化を図っている.
\par
%次へ進むボタンについて
また画面下部に結果出力機能へ遷移するボタンを配置してあるが,そのボタンが押下された時に全ての項目に対して各Providerの状態を確認するようになっている.
全ての項目が正しく入力,選択されている場合のみ結果表示画面へ遷移が可能となり,未入力のまま結果が出力されることを防いでいる.
\par

\subsection{結果出力機能}
%得点算出について
本機能は,利用者の回答内容から点数の合計とそのアドバイスを表示する機能と,各質問内容とその回答結果を一枚のチェックシート形式の画像として生成・保存する機能から構成されている.
まず点数の合計とそれに対するアドバイスを表示する機能では,ScoreResultPageクラスを作成した.
4.3.1節で述べたように,Flutter\_riverpodによって管理されているtrtScoreListProviderから各質問の点数を取得してその合計を算出する.
質問の個数と各点数配分から合計が45点満点であるため,利用者が直感的に達成度を理解できるよう,一般的な評価尺度として馴染み深い100点満点形式に換算して算出した.
そしてその総合得点を\_scoreAnimationとして0からカウントアップして表示される様にした.
他にも利用者の終活に対する意欲と達成感を向上させるため,総合得点を0〜59点,60〜79点,80点以上と分けて,それぞれの得点に応じた画像をアニメーションで表示させた.
また各分野での得点率を計算して,最も得点率が低かった分野を利用者が最も準備が必要な分野と判断し,それに対応した利用者が次に家族と話し合うべき具体的なテーマを\_adviceCommentとしてアニメーションを適用し画面に表示させた.
これにより利用者が自身の課題を認知し,それを改善する意欲を高められる様にした.
\par
%回答結果シートの出力について
回答結果のシートを生成,保存する機能ではPdfIfsheetPageクラスを用いた.
あらかじめ背景となる枠組みや質問内容を記載した画像を用意し,回答日の年月日や利用者の属性情報,回答内容の画像を重ね合わせることでシートを作成している.
各属性情報や結果内容の配置には,StackウィジェットとAlignウィジェットを用いた.
をselectionPositionsやdatePositionの様にあらかじめ各情報の座標を定義することで,
デバイスの画面サイズが異なる場合でも各ウィジェットの位置関係が崩れることなく回答結果のシートを出力することができる.
\par
%回答結果シートの保存について
回答結果のシートの画像保存にあたってはFlutter\_screenshotを用いて,画像保存ボタンを押すことでStackウィジェット全体を画像データとして取得するというScreenshotControllerインスタンスを作成した.
取得した画像データはFlutter\_path\_providerを利用することで,デバイスのアプリケーションの内部ストレージに保存することができる.
また保存する際のファイル名を,ファイルの保存場所/終活チェックシート\_回答年月日\_利用者のニックネーム.pngという形式で保存することにより,同一端末で同一人物が複数回回答をした際でも過去の回答結果を区別しやすい仕様となっている.
保存が完了したタイミングで'画像を保存しました'というSnackBarを表示して利用者に通知を行っている.


\subsection{履歴確認・共有機能}
%回答結果シートの一覧表示について
履歴確認機能の実装においては保存された画像の一覧を表示するRecordPageクラスと,選択した画像のプレビューを行うためのShowingomgPageクラスを作成した.
保存された画像の一覧情報の取得にはFlutter\_path\_providerを用いている.
RecordPageクラスの初期化時にアプリケーションのドキュメントディレクトリのパスを取得し,
取得したディレクトリ内から拡張子が「.png」であるファイルを選び出すことで,本アプリケーションで作成された回答結果シートの一覧リストを作成している.
また処理の実行中にはCircularProgressIndicatorを用いることで,画面の中央に円形のローディングアニメーションを表示させた.
これによりアプリケーションが停止しているのではなく,読み込み中であることを利用者に示すことができる.
%プレビュー画面について
一覧から各画像の表示にあたっての画面遷移にもgo\_routerを用いている.
表示する画像のPathParametersとしてimgPathを作成し,遷移先のShowimgPageクラスに引き渡すことで選択した画像を表示させることができる.
\par
%共有・削除機能のについて
また本機能では,各画像の一覧画面と選択した画像のプレビュー画面の両方から実行ができるようになっている.
共有機能ではRecActionクラス,削除機能ではCustomDialogクラスを作成しコンポーネント化することで,
一覧画面とプレビュー画面のどちらからでも正しく動作する挙動の一貫性と修正の効率化を図った.
共有機能においてはボタンの押下によってOS標準の共有シートを呼び出し,メールやAirDrop等で他端末に画像を共有することができる.

\section{UIの実装}
以下に本アプリケーションの画面遷移図を示す.
\begin{figure}[H]
      \centering
      \includegraphics[width=\textwidth]{images/chap4/mockup.png}
      \caption{画面遷移図}
      \label{fig:mockup}
\end{figure}

本アプリケーションの主要な対象者として,医療ソーシャルワーカーと通院頻度が高くソーシャルワーカーの支援を必要としやすい高齢者とその家族を想定している.
そのため高齢者の加齢に伴う視力の低下に配慮し,ボタンやテキストサイズは一般的なアプリケーションよりも大きく設計することで視認性を確保している.
操作ボタンについては,入力状況に応じて色を変化させることで操作の可否を直感的に判別可能にした.
また視覚的な補助として画像やピクトグラムを大きく配置し,テキストの判読が困難な場合でも操作意図が伝わるよう操作性の向上を図った.
